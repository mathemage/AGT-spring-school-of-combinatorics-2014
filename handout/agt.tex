\documentclass{article}
\usepackage{a4wide}
\usepackage[top=2cm, bottom=2cm, left=1.5cm, right=1.5cm]{geometry}
\usepackage{amsthm}
\usepackage{amssymb}

\newtheorem{theorem}{Theorem}
\newtheorem{proposition}{Proposition}
\newtheorem{lemma}{Lemma}
\newtheorem{corollary}{Corollary}
\newtheorem{fact}{Fact}
\newtheorem{claim}{Claim}
\theoremstyle{definition}
\newtheorem*{definition}{Definition}

\title{Algorithmic Game Theory -- Definitions \& Exercises}

\newcommand{\calM}{\mathcal{M}}
\newcommand{\calO}{\mathcal{O}}
\newcommand{\E}{\mathbf{E}}
\newcommand{\R}{\mathbb{R}}

\begin{document}
\maketitle


\section{Definitions}

\begin{itemize}
\item A game consists of a set $N$ of players $\{1, 2, \dots, n\}$.
Each player $i$ has his own set of possible (pure) strategies, say $S_i$.
To play the game, each player $i$ selects a strategy $s_i\in S_i$.

\item
We will use $s = (s_1, \dots, s_n)$ to denote the vector of strategies selected by the
players and $S = S_1\times\dots\times S_n$ to denote the set of all possible ways in which players can pick
strategies.

\item
The vector of strategies $s\in S$ selected by the players determine the \emph{outcome} for each player.
In a game, for each player it is given a preference ordering on these outcomes by
a complete, transitive, reflexive binary relation on the set of all strategy vectors.

\item A natural way to give the preference ordering on the outcomes is by defining
the payoff (utility) to each player $i$, or cost in other games, which are the functions
$u_i:S\to \R$, or $c_i:S\to \R$.

\item We say that a game is given in \emph{standard (matrix) form} if the list of all possible
strategy combinations together with respective payoffs is given.

%%%%%%%%%%%%%%%%%%%%%%%%% MERGE %%%%%%%%%%%%%%%%%%%%%%%%
\item
A strategy vector $s\in S$ is said to be a \emph{dominant strategy} if for each
player $i$ and each alternate strategy vector $s'\in  S$, we have that
$u_i(s_i, s'_{-i})\ge  u_i(s'_i, s'_{-i})$.
%%%%%%%%%%%%%%%%%%%%%%%%% MERGE %%%%%%%%%%%%%%%%%%%%%%%%

\item
A strategy vector $s\in S$ is said to be a \emph{pure Nash equilibrium} if for each player $i$ and each
alternate strategy $s'_i\in  S_i$, we have that $u_i(s_i, s_{-i})\ge  u_i(s'_i, s_{-i})$.

\item A \emph{mixed strategy $m_i:S_i\to[0,1]$} of player $i$ is a probabilistic distribution over the set $S_i$.
Each player $i$ thus aims to maximize the expected payoff $\=u_i(m_i, m_{-i})$ under this distribution.

\item A vector $m=(m_1,\dots,m_n)$ of mixed strategies of all players
is called \emph{mixed Nash equilibrium} if for each player $i$ and each
alternate mixed strategy $m'_i$ we have that $\=u_i(m_i,m_{-i})\ge \=u_i(m'_i,m_{-i})$.

\item A \emph{correlated equilibrium} is a probability distribution $p$ over
strategy vectors such that for each player $i$ and each alternate strategy
$s'_i$ we have that $\sum_{s_{-i}} p(s_i , s_{-i}) u_i(s_i , s_{-i}) \ge
\sum_{s_{-i}} p(s'_i , s_{-i}) u_i(s'_i , s_{-i})$ where $p(s) = p(s_i,
s_{-i})$ is the probability that an external correlation device suggests
a~strategy vector $s$.

\end{itemize}


\subsection{Exercises}

\begin{enumerate}

\item
Give a finite algorithm for finding a pure Nash equilibrium for a game with two players
defined in a standard form.
What about finding a mixed Nash equilibrium?

\item
Consider a two-player game given in standard form where each player has $n$ strategies.
Assume that the payoffs for each player are in the range $[0, 1]$ and are selected
independently and uniformly at random. Show that the probability that this random
game has a pure Nash equilibrium approaches $1 - 1/e$ as $n$ goes to infinity.

\item
We have seen that finding a Nash in a two-person zero-sum game is significantly
easier than general two-person games. Now consider a three-person zero-sum game,
that is, a game in which the rewards of the three players always sums to zero. Show
that finding a Nash equilibrium in such games is at least as hard as that in general
two-person games.

\item
Consider an $n$ person game in which each player has only two strategies.
A dependence graph $G$ is given, whose nodes are the players, and an edge between two
players $i$ and $j$ represents the fact that the payoff of player $i$ depends on the strategy
of player $j$ and vice versa. Thus, if node $i$ has $k$ neighbors in $G$,
then its payoff depends only on its own strategy and the strategies of its $k$ neighbors.
Assume that the graph $G$ is a tree with maximum degree 3.
Give a polynomial time algorithm to decide if such a game has a pure Nash
equilibrium.
(Recall that there are $2^n$ possible pure strategy vectors, yet your algorithm
must run in time polynomial in $n$.)

\item
Assume that $n$ companies, which produce the same product, are competing for
customers. If each company $i$ has a production level of $q_i$,
there will be $q = \sum_{i=1}^n q_i$ units of the product on the market.
Now, demand for this product depends on the price: if $q$ units are on the
market, price will settle so that all $q$ units are sold.
Assume that we are given a ``demand-price curve'' $p(d)$,
which gives the price at which all $d$ units can be sold.
Assume that $p(d)$ is a monotone decreasing, differentiable function of $d$.
With this definition, the income of the firm $i$ will be $q_i p(q)$.
Assume that production is very cheap and each firm will produce to maximize its income.
Show that for some $p(d)$ the total income for a monopolistic firm ($n=1$),
can be arbitrarily higher than the total income of many different firms sharing
the same market.
(Hint: You may try, e.g., $p(d) = 1 - d$.)

\end{enumerate}

% Mechanism design
\section{Introduction to mechanism design}
\subsection{Elections}
\begin{itemize}
\item Consider a set of alternatives $A$ (the candidates) and a set of $n$ voters I.
Let us denote by $L$ the set of linear orders on $A$.
The preferences of each voter $i$ are formally given by $\prec_i \in  L$,
where $a\prec_i b$ means that $i$ prefers alternative $b$ to alternative $a$.

\item
A function $f: L^n\to  A$ is called a \emph{social choice function}.
A function $F: L^n\to L$ is called a \emph{social welfare function}.

\item
A social welfare function $F$ satisfies \emph{unanimity} if for every $\prec\in L$,
$F(\prec,\dots,\prec)=\prec$.

\item
Voter $i$ is a \emph{dictator} in social welfare function $F$ if for all $\prec_1,\dots,\prec_n\in L$,
$F(\prec_1,\dots,\prec_n)=\prec_i$.
$F$ is not a dictatorship if no $i$ is a dictator in it.

\item
A social welfare function satisfies \emph{independence of irrelevant alternatives}
if the following holds:
For every $a, b\in A$ and every $\prec_1,\dots,\prec_n,\prec'_1,\dots,\prec'_n\in L$,
if we denote $\prec=F(\prec_1,\dots,\prec_n)$ and $\prec'=F(\prec'_1,\dots,\prec'_n)$,
then $a\prec_i b \Leftrightarrow a\prec'_i b$ for all $i$ implies that
$a\prec b \Leftrightarrow a\prec' b$.

\begin{theorem}[Arrow]
Every social welfare function over a set of more than 2 candidates ($|A| \geq 3$) that satisfies
unanimity and independence of irelevant alternatives is a dictatorship.
\end{theorem}

\begin{claim}[pairwise neutrality]
Let $>_1, /ldots, >_n$ and $>_1', /ldots, >_n'$ be two plazer profiles such that for everz player
$i$, $a >_i b \Leftrightarrow c >_i' d$. Then $a > b \Leftrightarrow c >' d$ where
$> = F(>_1, \ldots, >_n)$ and $>' = F(>_1', \ldots, >_n')$
\end{claim}

\item
A social choice function $f$ is monotone if
$f(\prec_1,\dots,\prec_i,\dots,\prec_n)=a\ne a'=f(\prec_1,\dots,\prec'_i,\dots,\prec_n)$
implies that $a'\prec_i a$ and $a\prec_i a'$.

\item
A social choice function $f$ can be \emph{strategically manipulated} by
voter $i$ if for some $\prec_1,\dots,\prec_n\in L$ and some $\prec'_i\in L$
we have that $a \prec_i a'$ where $a=f(\prec_1,\dots,\prec_i,\dots,\prec_n)$
and $a'=f(\prec_1,\dots,\prec'_i,\dots,\prec_n)$.
$f$ is called \emph{incentive compatible} if it cannot be manipulated.

\item Voter i is a \emph{dictator} in social choice function $f$ if for all
$<_1, \ldots <_n \in L, \forall b \neq a, a >_i b \Rightarrow F(<_1, /ldots, <_n) = a$
$f$ is called \emph{dictatorship} if some $i$ is a dictator in it.

\begin{theorem}
Let $f$ be an incentive complatible social choice function onto A, where $|A| \geq 3$, then $f$
is a dictatorship.
\end{theorem}

\end{itemize}

\subsection{Auctions}
\begin{itemize}

\item \emph{Vickrey second price auction:} Let the winner be the player $i$ with the highhest
declared value of $w_i$ and let $i$ pay the second highest declared bid $p^* = \max_{i \neq j} w_j$.

\begin{proposition}[Vickrey]
For everz $w_1, \ldots, w_n$ and every $w_i'$. Let $u_i$ be $i$'s utility if he bids $w_i$ and $u_i'$
his utilitz if he bid $w_i'$. Then $u_i \geq u_i'$.
\end{proposition}

\item $v_i: A \to R$ is a \emph{valuation function}, $v_i \in V_i$. $V_i \subseteq R^A$ is commonly
known set of possible valuation function for player $i$.

\item $v = (v_1, \ldots, v_n)$ -- $n$-dimensional vector. $v_{-i} = (v_1, \ldots, v_{i-1}, v_{i+1}, \ldots, v_n)$.
$(v_{i}, v_{-i}) = (v_1, \ldots, v_n)$. Similarly for $V, t, x, X$.

\item A \emph{mechanism} is a social choice function $f: V_1 \times \ldots \times V_n \to A$
and a vector of payment funcion s $p_1, \ldots, p_n$, where $p_i: V_1 \times \ldots \times V_n \to R$
is the amount that player $i$ pays.

\item A mechanism $(f, p_1, \ldots, p_n)$ is callen \emph{incentive compatible} if for every player $i$,
every $v_1 \in V_1, \ldots, v_n \in V_n$ and every $v_i \in V_i$, if we denote $a = f(v_i, v_{-i})$
and $a' = f(v_i', v_{-i}')$, then $v_i(a) - p_i(v_i, p_{-i}) \geq v_i(a') - p_i(v_i', v_{-i})$.

\item A mechanism $(f, p_1, \ldots, p_n)$ is callen \emph{Vickrey-Clarke-Groves (VCG) mechanism} if
\begin{itemize}
\item $f(v_1, \ldots, v_n) \in \texttt{argmax}_{a \in A} \sum_i v_i(a);$ that is, $f$ maximiyes the social welfare, and
\item for some functions $h_1, \ldots, h_n$, where $h_i: V_{-i} \to R$ (i. e. $h_i$ does not depend on $v_i$),
we have that for all $v_1 \in V_1, \ldots, v_n \in V_n: p_i(v_1, \ldots, v_n) = h_i(v_{-i})
- \sum_{j \neq i} v_i(f(v_1, \ldots, v_n))$. 
\end{itemize}

\begin{theorem}[Vickrey-Clarke-Grove]
Every VCG mechanism is incentive compatible.
\end{theorem}

\item A mechanism is \emph{individualy rational} if players always get nonnegative utility.
Formally if for every $v_1, \ldots, v_n$ we have that $v_i(f(v_1, \ldots, v_n)) - p_i(v_1, \ldots, v_n) \geq 0$.

\item A mechanism has no positive transfers if no players is ever paid money. Formally if for every $v_1, \ldots, v_n$
and every $i$, $p_i(v_1, \ldots, v_n) \geq 0$.

\item \emph{Clarke pivot rule:} The choice $h_i(v_{-i}) = \max \sum_{j \neq i} v_i(b)$ is called the Clarke pivot
payment. Under this rule the payment of player $i$ is $p_i(v_1, \ldots, v_n) = \max \sum_{j \neq i} v_j(b)
- \sum_{j \neq i} v_i(a)$, where $a = f(v_1, \ldots, v_n)$.

\begin{lemma}
A VCG mechanism with Clarke pivot payments makes no positive transfers. If $V_i(a) \geq 0$ for every $v_i \in V_i$
and $a \in A$ then it is also individually rational.
\end{lemma}

\end{itemize}

\subsection{Exercises}
\begin{enumerate}

\item Prove that a social choice function is incentive compatible if and only if it is monotone.

\item Show that a social welfare function that is a dictatorship satisfies
unanimity and independence of irrelevant alternatives.

\item Remind the last elections of president of Czech Republic. Do you think that first or second
or both rounds work as a dictatorship?

\item Is there any relation between Vickrey's Second Price Auction and Vickrey-Clarke-Grove mechanism
with a Clarke pivot rule?

\item There is one more exercise but that we tell you in a lecture.

\end{enumerate}

\section{Graphical Games}
For simplicity, each player $i$ has only two pure strategies $\{0, 1\}$.
A mixed strategy is given by a probability $m_i \in [0, 1]$ that the player will play 0.
We denote a vector of mixed strategies of all players by $\vec{m}$.

For an undirected graph $G$ on vertices $[n] := \{1, \dots n\}$ let $N(i)$ denote the
set of neighbors of $i$ and $i$ itself.

\begin{definition} A \textit{graphical game} is a pair $(G, \calM)$ where $G$ is an undirected graph on
vertices $[n]$ and $\calM$ is a set of $n$ \textit{local game matrices}.
For any vector of strategies $s$ the matrix $M_i \in \calM$ specifies the payoff $M_i(s)$
for the player $i$ which depends only on the actions taken by the players in $N(i)$.
\end{definition}

\subsection{Computing Nash Equilibria in Tree Graphical Games}
\begin{definition}[An approximation of Nash equilibrium] An \textit{$\varepsilon$-Nash equilibrium} ($\varepsilon$-NE) is a mixed
strategy $m$ such that for any player $i$ and each alternate mixed strategy $m'_i$
we have that $u_i(m_i,m_{-i}) + \varepsilon\ge u_i(m'_i,m_{-i})$ where $u_i(m_i,m_{-i})$ is
the expected payoff under the distribution $m$.
\end{definition}

\begin{theorem}
Let $(G, \calM)$ be a graphical game in which $G$ is a (rooted) tree with $n$ vertices
and let $d$ be the maximal number of children of a vertex. For any $\varepsilon > 0$
let $\tau = \calO(\varepsilon/d)$. Then there exists an algorithms which computes
an $\varepsilon$-Nash equilibrium for $(G, \calM)$ in a time polynomial
in $1/\varepsilon$, $n$ and $2^d$, while the representation of $(G, \calM)$ takes
$\calO(n2^d)$.
\end{theorem}

\subsection{Correlated Equilibria}

\begin{definition}
Two distributions $p$ and $q$ over vectors of strategies are \textit{expected payoff equivalent},
denoted $p \equiv_{\mathrm{EP}} q$, if $p$ and $q$ yield the same expected payoff vector, i.e.,
for each $i$ we have $\E_{s \sim p} u_i(s) = \E_{s \sim q} u_i(s)$ where $s \sim p$ means that $s$ is chosen
according to the distribution $p$.
\end{definition}

\begin{definition}
For any graph $G$, two distributions $p$ and $q$ over vectors of strategies are
\textit{local neighborhood equivalent} with respect to $G$, denoted $p \equiv_{\mathrm{LN}} q$,
if for all players $i$ and for all settings $s_{N(i)}$ of strategies of players in $N(i)$
we have $p(s_{N(i)}) = q(s_{N(i)})$.
\end{definition}

\begin{definition}
A local Markov network is a pair $M = (G, \Psi)$ where $G$ is an undirected graph on vertices $[n]$
and $\Psi$ is a set of \textit{potential functions}, one for each local neighborhood $N(i)$.
Each $\psi_i \in \Psi$ maps binary assignment of values of $N(i)$ to the range $[0,\infty)$,
i.e., $\Psi = \{ \psi_i: i \in [n]; \psi_i: \{ s_{N(i)} \} \rightarrow [0, \infty )\}$
where $s_{N(i)}$ is a set of $2^{|N(i)|}$ possible settings in $N(i)$.

A local Markow network $M$ defines a probability distribution $p_M$ as follows.
For any binary assignment $s$ to the vertices we define
$$p_M(s) := \frac{1}{Z}\left( \prod_{i=1}^n \psi_i (s_{N(i)}) \right)$$
where $Z = \sum_s \prod_{i=1}^n \psi_i (s_{N(i)}) > 0$ is a normalization factor.
\end{definition}

\begin{theorem}
For all graphical games $(G, \calM)$ and for any correlated equilibrium $p$
there exists a distribution $q$ such that
\begin{enumerate}
\item $q$ is also a correlated equilibrium for $(G, \calM)$,
\item $q$ gives all players the same expected payoff as $p$, i.e., $q \equiv_{\mathrm{EP}} p$,
\item $q$ can be represented as a local Markov network with the graph $G$.
\end{enumerate}

%Moreover, we can compute a correlated equilibrium of a tree graphical game with a linear program in a time
%polynomial $n$ and $2^d$.

\end{theorem}

\end{document}
